%%% DOCUMENTCLASS 
%%%-------------------------------------------------------------------------------

\documentclass[
a4paper, % Stock and paper size.
11pt, % Type size.
% article,
%oneside, 
onecolumn, % Only one column of text on a page.
%openright, % Each chapter will start on a recto page.
%openleft, % Each chapter will start on a verso page.
openany, % A chapter may start on either a recto or verso page.
%notitlepage % prvni stranku si udelam sama
oldfontcommands,
]{memoir}

%%% PACKAGES 
%%%------------------------------------------------------------------------------

\usepackage[utf8]{inputenc} % If utf8 encoding
% \usepackage[lantin1]{inputenc} % If not utf8 encoding, then this is probably the way to go
\usepackage[T1]{fontenc}    %
\usepackage[english]{babel} % English please
\usepackage[final,letterspace=-30]{microtype} % Less badboxes
% \usepackage{kpfonts} %Font
\usepackage{amsmath,amssymb,mathtools} % Math
% \usepackage{tikz} % Figures
\usepackage[final]{pdfpages} % insert PDF
\usepackage{epstopdf}
\usepackage{graphicx} % Include figures
\usepackage{aas_macros}
%\usepackage[round]{natbib}
\usepackage{multirow}

\usepackage{csquotes} % uvozovky nasledne prikaz \enquote{}
\usepackage{amsmath,amssymb,amsthm}
\usepackage{mathptmx}  %% volne dostupny font Adobe Times Roman
\usepackage{longtable}
\usepackage{dcolumn}% Align table columns on decimal point
\usepackage{bm}

\usepackage{xcolor}
\definecolor{ps_green}{RGB}{34,139,34}

\usepackage[square,numbers]{natbib} %nastavuje bibtex
%\bibliographystyle{plainnat} % seradi bib podle abecedy
\bibliographystyle{unsrt} %seřadí bib podle prvniho pouziti reference

%%% sci rep usepackage
\usepackage{calrsfs}
%\usepackage{physics}
%\usepackage{siunitx}
%\usepackage{breqn}
\usepackage{todonotes}
%%% end sci rep usepackage

\newsavebox{\foobox}
\newcommand{\slantbox}[2][0]{\mbox{%
		\sbox{\foobox}{#2}%
		\hskip\wd\foobox
		\pdfsave
		\pdfsetmatrix{1 0 #1 1}%
		\llap{\usebox{\foobox}}%
		\pdfrestore
}}
\newcommand\unslant[2][-.20]{\slantbox[#1]{$#2$}}


\newcommand\ualpha{\unslant\alpha}
\newcommand\sualpha{\scriptsize\unslant\alpha\kern-0.075em}
\newcommand\udelta{\unslant\delta}
\newcommand\upi{\unslant\pi}
\newcommand\umu{\unslant\mu}




%%% PAGE LAYOUT 
%%%------------------------------------------------------------------------------

%\setlrmarginsandblock{0.18\paperwidth}{*}{1} % Left and right margin % original: 0.15
%\setlrmarginsandblock{4.35cm}{3.35cm}{*} % rozlozeni na tisk -- v knihtisku orezou 11mm zleva a 1mm ze zbyvajicich stran
\setlrmarginsandblock{3.7cm}{2.7cm}{*} % rozlozeni na tisk -- v knihtisku orezou 11mm zleva a 1mm ze zbyvajicich stran
%\setulmarginsandblock{0.2\paperwidth}{*}{1}  % Upper and lower margin
\setulmarginsandblock{4.cm}{*}{1}  % Upper and lower margin
\checkandfixthelayout % obrátí okraje, když chceme oboustrannou sazbu, musí sudé stránky převrátit, při jednom stranné sazbě předchozí přímo nastaví okraje https://tex.stackexchange.com/questions/96446/set-different-margins-for-even-and-odd-pages

%%% SECTIONAL DIVISIONS
%%%------------------------------------------------------------------------------

\maxsecnumdepth{subsection} % Subsections (and higher) are numbered
\setsecnumdepth{subsection}

\makeatletter %
\makechapterstyle{standard}{
  \setlength{\beforechapskip}{0\baselineskip}
  \setlength{\midchapskip}{1\baselineskip}
  \setlength{\afterchapskip}{6\baselineskip} % mezera pod nadpisem kapitoly, original: 8
  \renewcommand{\chapterheadstart}{\vspace*{\beforechapskip}}
  \renewcommand{\chapnamefont}{\centering\normalfont\Large\scshape} % upravuje vzhled Chapter XX 
  \renewcommand{\printchaptername}{\chapnamefont \@chapapp}
  \renewcommand{\chapternamenum}{\space}
  \renewcommand{\chapnumfont}{\normalfont\Large\scshape} % upravuje vzhled nadpisu kapitol meho textu
  \renewcommand{\printchapternum}{\chapnumfont \thechapter}
  \renewcommand{\afterchapternum}{\par\nobreak\vskip \midchapskip \hrule \vskip\onelineskip}
  \renewcommand{\printchapternonum}{\vspace*{\midchapskip}\vspace*{5mm}}
  \renewcommand{\chaptitlefont}{\centering\bfseries\LARGE\scshape} % upravuje vzhled nadpisu typu Contents, Introduction
  \renewcommand{\printchaptertitle}[1]{\chaptitlefont ##1}
  \renewcommand{\afterchaptertitle}{\par\nobreak\vskip \afterchapskip}
  %\renewcommand{\afterchaptertitle}{\vskip\onelineskip \hrule\vskip\onelineskip}
  %\renewcommand*{\cftchapterdotsep}{\cftdotsep}
  %\renewcommand*{\cftsectionleader}{\hfill}
}
\makeatother

\chapterstyle{standard}

\setsecheadstyle{\normalfont\large\bfseries}
\setsubsecheadstyle{\normalfont\normalsize\bfseries}
\setparaheadstyle{\normalfont\normalsize\bfseries}
\setparaindent{0pt}\setafterparaskip{0pt}

%%% FLOATS AND CAPTIONS
%%%------------------------------------------------------------------------------

\makeatletter                  % You do not need to write [htpb] all the time
\renewcommand\fps@figure{htbp} %
\renewcommand\fps@table{htbp}  %
\makeatother                   %

\captiondelim{\space } % A space between caption name and text
\captionnamefont{\small\bfseries} % Font of the caption name
\captiontitlefont{\small\normalfont} % Font of the caption text

\changecaptionwidth          % Change the width of the caption
\captionwidth{1\textwidth} %

%%% ABSTRACT
%%%------------------------------------------------------------------------------

\renewcommand{\abstractnamefont}{\normalfont\small\bfseries} % Font of abstract title
\setlength{\absleftindent}{0.1\textwidth} % Width of abstract
\setlength{\absrightindent}{\absleftindent}

%%% HEADER AND FOOTER 
%%%------------------------------------------------------------------------------

\makepagestyle{standard} % Make standard pagestyle

\makeatletter                 % Define standard pagestyle
\makeevenfoot{standard}{}{}{} %
\makeoddfoot{standard}{}{}{}  %
%\makeevenhead{standard}{\bfseries\thepage\normalfont\qquad\leftmark}{}{}
\makeevenhead{standard}{\bfseries\thepage\mdseries\qquad\small\rightmark}{}{}
\makeoddhead{standard}{}{}{\small\rightmark\qquad\bfseries\thepage}
\makeheadrule{standard}{\textwidth}{\normalrulethickness} % cara nad textem na strance
\makeatother                  %

\makeatletter
\makepsmarks{standard}{
%\createmark{chapter}{both}{shownumber}{\@chapapp\ }{ \quad } % tohle pise do hlavicky stranek i "Chapter"
\createmark{chapter}{both}{shownumber}{}{ \quad }
\createmark{section}{right}{shownumber}{}{ \quad }
\createplainmark{toc}{both}{\contentsname}
\createplainmark{lof}{both}{\listfigurename}
\createplainmark{lot}{both}{\listtablename}
\createplainmark{bib}{both}{\bibname}
\createplainmark{index}{both}{\indexname}
\createplainmark{glossary}{both}{\glossaryname}
}
\makeatother                               %

\makepagestyle{chap} % Make new chapter pagestyle

\makeatletter
\makeevenfoot{chap}{}{\small\bfseries\thepage}{} % Define new chapter pagestyle
\makeoddfoot{chap}{}{\small\bfseries\thepage}{}  %
\makeevenhead{chap}{}{}{}   %
\makeoddhead{chap}{}{}{}    %
% \makeheadrule{chap}{\textwidth}{\normalrulethickness}
\makeatother

\nouppercaseheads
\pagestyle{standard}               % Choosing pagestyle and chapter pagestyle
\aliaspagestyle{chapter}{chap} %

%%% NEW COMMANDS
%%%------------------------------------------------------------------------------

\newcommand{\p}{\partial} %Partial
% Or what ever you want
\defcitealias{XMM-handbook}{XMM-Newton Users Handbook}
\newcolumntype{C}{>{\centering\arraybackslash}X}

%%% TABLE OF CONTENTS
%%%------------------------------------------------------------------------------

\maxtocdepth{subsection} % Only parts, chapters and sections in the table of contents
\settocdepth{subsection}

\makeatletter
%\renewcommand{\@tocrmarg}{2.55em plus1fil}
\makeatother

\AtEndDocument{\addtocontents{toc}{\par}} % Add a \par to the end of the TOC

%%% INTERNAL HYPERLINKS
%%%-------------------------------------------------------------------------------

\usepackage{hyperref}   % Internal hyperlinks
\hypersetup{
pdfborder={0 0 0},      % No borders around internal hyperlinks
pdfauthor={Petr Steindl} % author
}
\usepackage{memhfixc}   %

%%% THE DOCUMENT
%%% Where all the important stuff is included!
%%%-------------------------------------------------------------------------------

\author{Petr Steindl}
\title{Investigation of the influence of the chemical composition of semiconductor quantum dots on their electronic structure}

\usepackage{lipsum} % Just to put in some text

\graphicspath{{../figures/}} % nastaveni cesty ke slozce s obrazky
\usepackage{multirow} % kvuli logum na titulni strane
\usepackage{booktabs} % kvuli care pod prostrednim radkem tabulky s logy na krajich
\usepackage{subcaption}


\def\changemargin#1#2{\list{}{\rightmargin#2\leftmargin#1}\item[]}
\let\endchangemargin=\endlist 

\begin{document}

\frontmatter
\pagestyle{empty} % bez hlavicky a paticky, tj. nevypise cislo strany
\savepagenumber % cislo z teto stranky si zapamatuje...

%\begin{center}
%	\ \\
%	\LARGE
%	MASARYKOVA UNIVERZITA\\
%	\scshape\Large Přírodovědecká fakulta\\
%			\large Ústav fyziky kondenzovaných látek\\
%	
%	\vspace{80mm}
%
%	{\Large DIPLOMOVÁ PRÁCE}
%	
%	\vspace{80mm}
%\end{center}
%\vfill
%Brno 2018 \hfill Petr Steindl
	
%\newpage

\restorepagenumber % ...a vlozi ho sem a pokracuje v cislovani. Titulni strana ma mit cislo i (1).

%\begingroup
%\setlength{\tabcolsep}{3pt}
%	\begin{center}
%	\centering
%	\begin{tabular}{rcl}
%		\multirow{5}{*}{\includegraphics[height=2.8cm]{MUNI_logo.png}} & &
%		\multirow{5}{*}{\includegraphics[height=2.8cm]{PrF_logo2.png}}\\[-0.19cm]
		%\multirow{5}{*}{\includegraphics[height=2.8cm]{MUNI_logo.eps}} & &
		%\multirow{5}{*}{\includegraphics[height=2.8cm]{PrF_logo2.eps}}\\[-0.19cm]
%		& \lsstyle\bfseries\Large {MASARYKOVA UNIVERZITA}&\\[0.33cm]
		%& \lsstyle\bfseries\Large {MASARYK UNIVERSITY}&\\[0.33cm]
%		& \lsstyle\bfseries\Large\scshape {Přírodovědecká fakulta} & \\[0.25cm]
		%& \lsstyle\bfseries\Large\scshape {Faculty of Science} & \\[0.25cm]
%		\lsstyle\bfseries\scshape & \small\lsstyle\bfseries\scshape {Ústav fyziky kondenzovaných látek} & \lsstyle\bfseries\scshape \\ 
		%\lsstyle\bfseries\scshape & \small\lsstyle\bfseries\scshape {Department of Condensed Matter Physics} & \lsstyle\bfseries\scshape \\ 
%		& & \\ [-0.3cm] \cmidrule[2pt]{2-2}
%	\end{tabular}
%\end{center}
%\endgroup

%\vspace{5cm}
%\begingroup
%\noindent %\huge{\textbf{Investigation of the influence of the chemical composition of semiconductor quantum dots on their electronic structure}}
%\huge{\textbf{Studium vlivu složení polovodičových\\
%		 kvantových teček na jejich elektronovou \\
%		 strukturu}}
%\vspace{0.5cm}\\
%\Large{Diploma Thesis}
%\Large{Diplomová práce}
%\vspace{0.5cm}\\
%\huge{\bfseries{Petr Steindl}}
%\vfill
%\noindent\normalsize\bfseries{Vedoucí práce: Mgr. Petr Klenovský, Ph.D. \hfill Brno 2018}
%\noindent\normalsize\bfseries{Supervisor: Mgr. Petr Klenovský, Ph.D. \hfill Brno 2018}
%\noindent\normalsize\bfseries{Vedoucí práce: Mgr. Petr Klenovský, Ph.D. \hfill Brno 2018}
%\endgroup

%\newpage
%\cleardoublepage
%\Large\textbf{Bibliografický záznam}\\

%\normalsize
%\renewcommand{\arraystretch}{2}
%\begin{tabular}{ll}
%	\textbf{Autor:} & Petr Steindl\\[-0.5cm]
%	& Přírodovědecká fakulta, Masarykova univerzita\\[-0.5cm]
%	& Ústav fyziky kondenzovaných látek\\
%	\textbf{Název práce:} & Studium vlivu složení polovodičových kvantových teček \\[-0.5cm]
%	& na jejich elektronovou strukturu\\ 
%	\textbf{Studijní program:}& Fyzika \\
%	\textbf{Obor:}& Fyzika kondenzovaných látek \\
%	\textbf{Vedoucí práce:}& Mgr. Petr Klenovský, Ph.D. \\
%	\textbf{Akademický rok:}& 2017/2018 \\
%	\textbf{Počet stran:}& xxii + 94 \\
%	\textbf{Klíčová slova:}& kvantové tečky; typ-II; modrý posuv; fotoluminiscence; \\[-0.5cm]
%	&časově rozlišená fotoluminiscence; elektronová a excitonová struktura;\\[-0.5cm]
%	& elektrický dipólový moment; InAs; GaAs; GaP \\
%\end{tabular} 


%\newpage
%\cleardoublepage
%\Large\textbf{Bibliographic record}\\

%\normalsize
%\begin{tabular}{ll}
%	\textbf{Author:} & Petr Steindl \\[-0.5cm]
%	& Faculty of Science, Masaryk University \\[-0.5cm]
%	& Department of Condensed Matter Physics \\[-0.5cm]
%	\textbf{Title of Thesis:} &  Investigation of the influence of the chemical composition \\[-0.5cm]
%	& of semiconductor quantum dots on their electronic structure\\
%	\textbf{Degree Programme:}& Physics \\
%	\textbf{Field of study:}& Condensed Matter Physics \\
%	\textbf{Supervisor:}& Mgr. Petr Klenovský, Ph.D. \\
%	\textbf{Academic Year:}& 2017/2018 \\
%	\textbf{Number of Pages:}& xxii + 94 \\
%	\textbf{Keywords:}& quantum dots; type-II; blue-shift; photoluminescence; \\[-0.5cm]
%	&time-resolved photoluminescence; electronic and excitonic structure;\\[-0.5cm]
%	& electron-hole dipole moment; InAs; GaAs; GaP \\
%\end{tabular}
%\renewcommand{\arraystretch}{1} 
%\newpage

%\cleardoublepage

%%%%%%%%%%%%%%%%%%%%%%%%%%%%%%%%%%%%%%%%
%%%%%%%%% ABSTRAKT - CZ
%\noindent\Large\textbf{Abstrakt}\\ \normalsize

%\noindent Tato práce se zaměřuje na experimentální a teoretické studium kvantových teček (QDs) připrave\-ných z III--V polovodičů epitaxí z organokovových sloučenin. Pro studium byly vybrány tři systémy materiálů: (i) napětím laděné InGaAs QDs, (ii) InAs QDs překryté GaAsSb vrstvou a~(iii) In$_{1-x}$Ga$_x$As$_y$Sb$_{1-y}$ QDs na GaP substrátu s 5~ML GaAs mezivrstvou. Elektronová stuktura těchto systémů byla počítána pomocí aproximace obálkových funkcí založené na 8-pásové $\mathbf{k\cdot p}$ teorii, multičásticové korekce byly zohledněny metodou konfigurační interakce. U systémů (ii) a~(iii) byla elektronová struktura navíc studována pomocí fotoluminiscenční (PL) spektroskopie. 


%Pro QDs vyrobené z III--V polovodičů bylo zjištěno, že nelineární členy piezoelektrického jevu dominují nad lineárními. Přestože energii lokalizovaných stavů započtení těchto členů ovlivňuje jen minimálně, pro tečky externě laděné aplikovaným anizotropním napětím byly již dříve prokázány dramatické změny elektrického dipólového momentu, které mohou být popsány pouze prvky druhého řádu piezoelektrického tenzoru. Provedli jsme tuto analýzu a naše výpočty ukazují, že dominantní roli na změnu dipólu hrají předpětí způsobené připevněním vzorku na stolek s piezo-pohonem a výška teček. Z toho důvodu jsme vyvinuli analytické modely pro odhadnutí změn dipólu jako funkce výšky teček a/nebo předpětí poskytující experimentátorům vodítko k naladění dipólu.


%Pomocí intenzitně a polarizačně rozlišené PL spektroskopie byla studována excitonová struktura InAs QDs, které díky tlusté krycí GaAsSb vrstvě mají uvěznění typu-II, s velkým modrým posuvem. Srovnáním PL spekter s vícečásticovými výpočty metodou konfigurační interakce byly identifikovány vícečásticové přechody neutrálního excitonu, biexcitonu a negativního trionu. Modrý posuv těchto vícečásticových přechodů byl srovnán s výpočty pomocí konfigurační interakce se self-konzistentním cyklem, které poskytují uspokojivou shodu.


%Na závěr studujeme systémy In$_{1-x}$Ga$_x$As$_y$Sb$_{1-y}$/GaP a In$_{1-x}$Ga$_x$As$_y$Sb$_{1-y}$/GaAs/GaP QD, které byly navrženy pro tzv.~\enquote{QD flash} paměťové jednotky. Ze studia InGaAs QDs je dobře známo, že přidání Sb do struktury často vede k uvěznění typu-II a ke zvýšení délky rekombinace. Proto jsme zkoumali vliv Sb na tento systém, tj. na excitonovou strukturu a uspořádání pásů v~reálném i reciprokém prostoru In$ _ {1-x}$Ga$ _x $As$ _y $Sb$ _ {1-y} $/GaAs/GaP QD s použitím celé řady metod od fotoluminiscence ustálených stavů po časově rozlišenou PL s podporou vícečásticových výpočtů.

%\noindent Tato práce se zaměřuje na experimentální a teoretické studium kvantových teček (QDs) připrave\-ných z III--V polovodičů epitaxí z organokovových sloučenin. Pro studium byly vybrány tři systémy materiálů: (i) napětím laděné InGaAs QDs, (ii) InAs QDs překryté GaAsSb vrstvou a~(iii) InGaAsSb QDs na GaP substrátu s 5~ML GaAs mezivrstvou. Elektronová stuktura těchto systémů byla počítána pomocí aproximace obálkových funkcí založené na 8-pásové $\mathbf{k\cdot p}$ teorii, multičásticové korekce byly zohledněny metodou konfigurační interakce (CI). U systémů (ii) a~(iii) byla elektronová struktura navíc studována pomocí fotoluminiscenční (PL) spektroskopie. 

%Pro QDs vyrobené z III--V polovodičů bylo zjištěno, že nelineární členy piezoelektrického jevu dominují nad lineárními. Přestože energii lokalizovaných stavů započtení těchto členů ovlivňuje jen minimálně, pro tečky externě laděné aplikovaným anizotropním napětím byly již dříve prokázány dramatické změny elektrického dipólového momentu, které mohou být popsány pouze prvky druhého řádu piezoelektrického tenzoru. Provedli jsme tuto analýzu a naše výpočty ukazují, že dominantní roli na změnu dipólu hrají předpětí způsobené připevněním vzorku na stolek s piezo-pohonem a výška teček. Z toho důvodu jsme vyvinuli analytické modely pro odhadnutí změn dipólu jako funkce výšky teček a/nebo předpětí poskytující experimentátorům vodítko k naladění dipólu.


%Pomocí intenzitně a polarizačně rozlišené PL spektroskopie byla studována excitonová struktura InAs QDs, které díky tlusté krycí GaAsSb vrstvě mají uvěznění typu-II, s velkým modrým posuvem. Srovnáním PL spekter s vícečásticovými výpočty byly identifikovány přechody neutrálního excitonu, biexcitonu a negativního trionu. Modrý posuv těchto přechodů byl srovnán s výpočty pomocí CI se self-konzistentním cyklem, které poskytují uspokojivou shodu.


%Na závěr studujeme systémy InGaAsSb/GaAs/GaP QD, které byly navrženy pro tzv.~\enquote{QD flash} paměťové jednotky. Ze studia InGaAs QDs je dobře známo, že přidání Sb do struktury často vede k uvěznění typu-II a ke zvýšení délky rekombinace. Proto jsme zkoumali vliv Sb na tento systém, tj. na excitonovou strukturu a uspořádání pásů v~reálném i reciprokém prostoru InGaAsSb/GaAs/GaP QD pomocí PL spektroskopie s podporou vícečásticových výpočtů.




%\vspace{3cm}
%\vfill

%\cleardoublepage
%%%%%%%%%%%%%%%%%%%%%%%%%%%%%%%%%%%%%%%%
%%%%%%%%% ABSTRACT - EN
\noindent\Large\textbf{Abstract}\\ \normalsize

\noindent This thesis deals with experimental and theoretical investigations of quantum dots (QDs) fabricated from III--V semiconductors by metalorganic vapour phase epitaxy. Three material systems are considered,~i.~e., (i) stress tuned InGaAs QDs, (ii) InAs QDs capped by GaAsSb layer, and (iii) In$_{1-x}$Ga$_x$As$_y$Sb$_{1-y}$ QDs on GaP substrate with GaAs interlayer. 
The electronic structure of the systems is calculated using the envelope function approach based on 8-band $\mathbf{k\cdot p}$ theory with multi-particle corrections considered within the configuration interaction method. Apart from that, for systems (ii) and (iii) we investigate the structures also by the photoluminescence (PL) spectroscopy.

The non-linear piezoelectricity in III--V semiconductor QDs was found to be dominant compared to the linear one. Even though the influence of the former on the energy of localised states in QDs is rather small, it was shown previously that dramatic changes of electric dipole moment for dots externally tuned by applied anisotropic stress can only be described by second-order elements of the piezo-electric tensor. 
%
%To proper modelling of the electronic structure of strain tuned InGaAs QDs have to be added high build-in in-plane prestress as an effect of bonding sample onto piezo actuator. 
%
We proceed with that analysis and our calculations show that the prestress caused by bonding of the sample onto piezo actuator and dot height play a dominant role for dipole change. Motivated by that, we develop analytical models for estimation of dipole change as a function of dot height and/or prestress to provide guidance for experimentalists.

Furthermore, the excitonic structure of InAs QDs with GaAsSb capping layer is studied. This system presents both type-I and type-II band-alignment. A large blue-shift with increasing pumping, characteristic for type-II, we study by intensity and polarization resolved PL. By comparing the results with multi-particle calculations performed by configuration interaction method we identify multi-particle optical transitions in PL spectra. Those are neutral exciton, biexciton, and negative trion. On the other hand, the measured blue-shift of the transitions is compared to self-consistent multi-particle calculations providing satisfactory compliance.

%For In$_{1-x}$Ga$_x$As$_y$Sb$_{1-y}$/GaP QDs elsewhere in literature indirect optical transitions have been discussed. 
%Indirect optical transitions for In$_{1-x}$Ga$_x$As/GaP QDs were discussed elsewhere in literature in the past. The indirect transitions resulting in long recombination time which could be used for long storage time devices as QD-Flash memories. 
%
Finally, we study In$_{1-x}$Ga$_x$As$_y$Sb$_{1-y}$/GaAs/GaP QD systems that were proposed as building blocks for novel so-called \enquote{QD flash} memories.
It is well known from studies of InGaAs QDs that adding Sb into the structure often leads to type-II band-alignment and increase of recombination lifetime. We have, thus, tested the influence of Sb on the latter system, i.~e., on the excitonic structure and band-alignment in both real and reciprocal spaces of In$_{1-x}$Ga$_x$As$_y$Sb$_{1-y}$/GaAs/GaP QDs using a variety of methods ranging from steady-state to time-resolved PL, supporting the explanation of those results with multi-particle theory.
%%%%%%%%%%%%%%%%%%%%%%%%%%%%%%%%%%%%%%%%
%%%%%%%%% INPUT ZADANI - PDF
%\newpage
%\includepdf[width=1.3\textwidth]{ofic_zadani_bezpodpisu.pdf}
%\cleardoublepage
%\includepdf[width=1.3\textwidth]{ofic_zadani_podpis.pdf}
%\newpage


%%%%%%%%%%%%%%%%%%%%%%%%%%%%%%%%%%%%%%%%
%%%%%%%%% ACKNOWLEDGEMENTS
\clearpage
\noindent\Large\textbf{Acknowledgements}\\ \normalsize

\noindent I would like to express my sincere gratitude to my supervisor Dr. Petr Klenovský for the continuous support during my master study and related research, for his motivation and guidance throughout this work, including introduction to the theoretical background of the problem.

Besides my supervisor, many thanks belong to Prof. Thomas Fromherz for a kind welcome at Johannes Kepler University in Linz, for support during my stay there and discussion around my theoretical results. I would like to thank also the group of Prof. Rinaldo Trotta, namely B.Sc. Johannes Aberl, for sharing their experimental results that I needed for my theoretical description.

It is an honour for me to collaborate with Dr. Benito Millán Alén from Instituto de Micro~y~Nanotecnología of Spanish National Research Council. Under his leadership and in his laboratory in Madrid, I have performed the measurements on InGaAsSb/GaP quantum dots.


This thesis would not have been possible without proper samples, hence I am particularly grateful for their growth by Dr. Alice Hospodková from the Institute of Physics of the Czech Academy of Science and by M.Sc. Elisa Maddalena Sala from the group of Prof. Dieter Bimberg from Technische Universität Berlin.

I also thank Ing. Jan Michalička for assistance during TEM measurements at CEITEC Nano, all colleagues at Department of Condensed Matter Physics for discussions about physics.


%%%%%%%%%%%%%%%%%%%%%%%%%%%%%%%%%%%%%%%%
%%%%%%%%% PROHLÁŠENÍ
\vfill
\noindent\Large\textbf{Prohlášení}\\ \normalsize

\noindent Prohlašuji, že jsem svoji diplomovou práci vypracoval samostatně
s využitím informačních zdrojů, které jsou v práci citovány.
\vspace{1cm}
\begin{center}
	\centering
	\begin{tabular}{p{0.5\linewidth}p{0.15\linewidth}p{0.25\linewidth}}
		Brno 16.\,května\,2018 &  & \\\cmidrule[0.5pt]{3-3}
		&&\centering Podpis autora \\ 
	\end{tabular}
\end{center}

\newpage
\cleardoublepage
%\begin{abstract}
%\lipsum[1-2]
%k tomuto textu je tu poznamka pod carou.\footnote{poznamka pod carou}
%\end{abstract}
%\clearpage


\pagestyle{standard}

\tableofcontents*


\clearpage



%%%%%%%%%%%%%%%%%%%%%%%%%%%%%%%%%%%%%%%%
%%%%%%%%% INTRODUCTION
%\chapter{Introduction}\label{chap:introduction}


%Since 1980 when Russian physicist Ekimov first observed quantum dots (QDs) on a glass crystal~\cite{Ekimov}, the research in this direction has been vastly expanding. This is due to the fact that QDs enable the spatial limitation of movement of electrons, thus, exhibit properties similar to atoms which, however, depend on their shape and size. In 1984, the link between the size of the semiconductor nanoparticle and its band-gap was derived by Luis Brus~\cite{Brus} which triggered the decade of investigation of QDs ending with the successful synthesis of colloidal CDX (X = S, Se, Te) QDs with a quantifiable absorption edge~\cite{Murray}. Since their discovery, QDs have been studied mainly for a variety of optical applications. Since 1990s, the studies have been focused on QDs fabricated from materials from III-V groups of the periodic table, which promised usage in electronic devices and computer technology, e.~g., the replacement of electrical circuits by optical ones~\cite{Bimberg}. %Historical details can be found eg in \cite{Bimberg}.

%\section{Quantum confinement effect}
%Semiconductor QDs are islands of one kind of semiconductor material of the size of nanometres or tens of nanometres in all three spatial directions embedded in the matrix of a different one. Because the size of these structures are comparable to or smaller than the de Broglie wavelength of electron written as
%\begin{equation}
%\lambda=\frac{h}{\sqrt{3m^*k_\mathrm{B}T}},
%\end{equation}
%where $m^*$ is the effective mass, $h$ and $k_\mathrm{B}$ are the Planck and the Boltzmann constants, respectively, and $T$ is temperature, quantization effects begin to play an essential role of the QD properties.




\subsection*{Outline of the thesis}
\vspace{0.2cm}
In this work, we study effects of the material composition of III--V semiconductor QDs on their electron structure. Because this thesis connects three individual topics dealing with those QDs, we summarise each of them separately in the end of the appropriate chapter.

In this shorten version of the thesis 
%
%In Chapter~\ref{chap:theory}, we demonstrate theoretical background of several approximations leading to the Envelope function theory usually used to simulate electronic structure of QDs and also the multiparticle theory correction treated within the Configuration interaction method.
%
%In Chapter~\ref{chap:2order_piezo} we investigate the effect of second order piezoelectricity on electronic %and excitonic
%structure of stress-tuned InGaAs/GaAs QDs using the eight-band $\mathbf{k \cdot p}$ and the Configuration interaction methods.
%
%Chapter~\ref{chap:SciRep} is focused on the experimental study of excitonic structure of type-II InAs QDs on GaAs substrate capped by GaAsSb layer by performing photoluminescence measurements and the observed blue-shift of emission energies with laser pumping occuring for these structures is examined. Both excitonic structure and blue-shift is supported by multi-particle calculations.
%
Chapter~\ref{chap:TUB_QD} presents experimental study of InGaAsSb/GaAs/GaP QDs which is one of the promising candidates for long storage time and application in QD-Flash memories.  

Finally, the thesis is summarised in Chapter~\ref{chap:summary}.
\newpage 




\mainmatter

%%%%%%%%%%%%%%%%%%%%%%%%%%%%%%%%%%%%%%%%
%%%%%%%%% INPUT CHAPTERS
%%%%%%%%%%%%%%%%%%%%%%%%%%%%%%%%%%%%%%%%%%%%%%%%%%%%%%%%%%%%%%%%%%%%
%\input{introduction/introduction.tex}
%\input{Chapter1_theory/dip_ch1_theory_pk_aj.tex}
%\input{Chapter2_piezo/dip_ch2_piezo.tex}

\cleardoublepage
%\input{Chapter4_SciRep/dip_ch4_scirep_aj.tex}

%\cleardoublepage
\input{Chapter3_berlin/dip_ch3_berlin_3_typIvsII_TRPL.tex}
\input{Thesis_summary/Thesis_summary.tex}

%%% -------------------------------------------------------------
%%%%%%%%%%%%%%%%%%%%%%%%%%%%%%%%%%%%%%%%
%%%%%%%%% INPUT APPENDIX
\cleardoublepage
\appendix
%\chapter{Material parameters used in 8-band $\bf{k \cdot p}$ calculations of InGaAs QDs} \label{app:material_params}

\begin{table*}[!ht]
	\caption{Values of the material parameters used in the calculations. The labeling is defined in Tab.~\ref{tDesc} and the references from which the parameters were taken are identified in the last column.\label{tSb2} 
		}
	\begin{center}
		\begin{tabular}{llccc}
			\hline \hline
			Parameter & Unit & InAs & GaAs & Ref.\\
			\hline
			$a$ & \AA & $6.0583$ & $5.6533$  & \cite{Vurgaftman}\\
			$a_{\rm exp}$ & \AA /K& $2.74\times$10$^{-5}$ & $3.88\times$10$^{-5}$  & \cite{Vurgaftman}\\
			$C_{11}$ & GPa& $83.29$ & $122.1$  & \cite{Vurgaftman}\\
			$C_{12}$ & GPa& $45.26$ & $56.6$  & \cite{Vurgaftman}\\
			$C_{44}$ & GPa& $39.59$ & $60.0$ & \cite{Vurgaftman}\\
			$E_0$ & eV & $0.417$ & $1.519$ &  \cite{Vurgaftman}\\
			$\varepsilon_{\rm r}$ & -& $15.15$ & $12.93$ &\cite{landoltbornstein}\\
			$\alpha$ & eV/K & $0.276\times$10$^{-3}$ & $0.541\times$10$^{-3}$ &  \cite{Vurgaftman}\\
			$\beta$ & K & $93$ & $204$ & \cite{Vurgaftman}\\
			$E_{\rm v}$ & eV & $1.390$ & $1.346$ &  \cite{zunger}\\
			$\Delta_0$ & eV & $0.390$ & $0.341$ &  \cite{Vurgaftman}\\
			$a_{\rm c}$ & eV & $-6.66$ & $-9.36$ & \cite{zunger}\\
			$a_{\rm v}$ & eV & $-1.00$ & $-1.21$ & \cite{zunger}\\
			$a_{\rm ub}$ & eV & $-1.8$& $-2.0$&  \cite{Vurgaftman}\\
			$a_{\rm ud}$ & eV & $-3.6$& $-4.8$&  \cite{Vurgaftman}\\
			$S$ & -& $-4.80$ & $-2.88$ & \cite{Vurgaftman}\\
			$E_{\rm p}$ & eV & $21.5$ & $28.8$ & \cite{Vurgaftman}\\
			$L$ & $\hbar^2/2m_0$ & $-15.695$ & $1.420$ & \cite{Vurgaftman}\\
			$M$ & $\hbar^2/2m_0$ & $-4.0$ & $-3.9$ & \cite{Vurgaftman}\\
			$N$ & $\hbar^2/2m_0$ & $-15.895$ & $0.056$ & \cite{Vurgaftman}\\
			\hline \hline
		\end{tabular}
	\end{center}
\end{table*}

% InAs  8x8kp-parameters                    = -10.4627d0 -4.0d0   -10.6627d0 ! L',M,N' [hbar^2/2m] (--> divide by hbar^2/2m)
%\newpage

\begin{table*}[!ht]
	\caption{Composition dependence of the input parameters of $\mathrm{In_xGa_{1-x}As}$ used in the calculations. The labeling of parameters is defined in Tab.~\ref{tDesc}. The references from which parameters were taken are identified in the last column. Those parameters whose reference is missing were provided by the parameter library of nextnano$^3$~\cite{next}.\label{tSb3}
	}
	\begin{center}
		\begin{tabular}{llcc}
			\hline \hline
			Parameter & Unit & $\mathrm{In_xGa_{1-x}As}$ & Ref.\\
			\hline
			$a$ & \AA & linear & \\
			$a_{\rm exp}$ & \AA /K & linear & \\
			$C_{11}$ & GPa& linear &  \\
			$C_{12}$ & GPa& linear &  \\
			$C_{44}$ & GPa& linear & \\
			%$e_{14}$ & C/m$^2$ & linear & \\
			$E_0$ & eV & $0.417x+1.519(1-x)-0.477x(1-x)$ & \cite{Vurgaftman}\\
			$\varepsilon_{\rm r}$ & -& linear & \\
			$\alpha$ & eV/K & linear & \\
			$\beta$ & K & linear & \\
			$E_{\rm v}$ & eV & $1.39x+1.346(1-x)+0.38x(1-x)$& \cite{Vurgaftman}\\
			$\Delta_0$ & eV & $0.39x+0.341(1-x)-0.15x(1-x)$& \cite{Vurgaftman}\\
			$a_{\rm c}$ & eV & $-6.66x-9.36(1-x)-2.61x(1-x)$& \cite{Vurgaftman}\\
			$a_{\rm v}$ & eV & linear & \\
			$a_{\rm ub}$ & eV & linear& \\
			$a_{\rm ud}$ & eV & linear& \\
			$S$ & -& $-4.80x-2.88(1-x)-3.54x(1-x)$& \cite{Vurgaftman}\\
			$E_{\rm p}$ & eV & $21.5x+28.8(1-x)+1.48x(1-x)$& \cite{Vurgaftman}\\
			$L$ & $\hbar^2/2m_0$ & $-15.695x+1.420(1-x)+25.063x(1-x)$& \cite{Vurgaftman}\\
			$M$ & $\hbar^2/2m_0$ & $-4.0x-3.86(1-x)+1.141x(1-x)$& \cite{Vurgaftman}\\
			$N$ & $\hbar^2/2m_0$ & $-15.895x+0.056(1-x)+26.809x(1-x)$& \cite{Vurgaftman}\\
			\hline \hline
		\end{tabular}
	\end{center}
\end{table*}

\newpage 
%\input{Apendix/Appendix1_principal_strains.tex}
%\input{Apendix/Appendix_strain_tune_fits.tex}
%\input{Apendix/Appendix4_SciRep.tex}
\input{Apendix/Appendix2_TEM.tex}
\input{Apendix/Appendix3_TRPL_temp.tex}

%\chapter{Causality}

%\lipsum[1-2]





\backmatter

%%% BIBLIOGRAPHY
%%% -------------------------------------------------------------


%\bibliography{example2}
%\bibliographystyle{plainnat}
%\bibliographystyle{alpha}
\bibliographystyle{unsrt}
%\bibliography{dip_ps_ref}
\bibliography{dip_ps_ref_final}



%%% -------------------------------------------------------------
%publications
%\input{publications/publications.tex}
%\chapter{List of publications}\label{chap:publications}

%\addcontentsline{toc}{chapter}{\bibname}
%\section*{Published papers}
%\begin{itemize}
%	\item P. Klenovský, D. Hemzal, P. Steindl, M. Zíková, V. Křápek and J. Humlíček, \textit{Polarization anisotropy of the emission from type-II quantum dots}, Phys. Rev. B \textbf{92}, 241302 (2015).
	
%	\item P. Klenovský, P. Steindl, D. Geffroy, \textit{Excitonic structure and pumping power dependent emission blue-shift of type-II quantum dots}, Scientific Reports \textbf{7}, 45568 (2017).
	
%\item P. Klenovský, P. Steindl, J. Aberl, E. Zallo, R. Trotta, A. Rastelli, T. Fromherz, \textit{Effect of second order piezoelectricity on excitonic structure of stess-tuned InGaAs/GaAs quantum dots}. Phys. Rev. B. Manuscript submitted
%\end{itemize}

%\section*{Poster presentations}
%\begin{itemize}
%	\item[1.]  P. Klenovský, D. Hemzal, P. Steindl, M. Zíková, V. Křápek and J. Humlíček, \textit{Polarization anisotropy of the emission from type-II quantum dots}, TWINFUSYON summerschool Advanced School on Modelling and Statistics for Biology, Biochemistry and Biosensing, Linz (Austria), 2016. 
	
	
%	\item[2.] P. Steindl, P. Klenovský, \textit{Excitonic structure and pumping power dependent emission blue-shift of type-II quantum dots}, TWINFUSYON winterschool New Frontiers in 2D materials: Approaches \& Applications, Villard-de-Lans (France), 2017.
	
%	\item[3.] P. Steindl, P. Klenovský, D. Geffroy, \textit{Multi-excitonic structure of type-II quantum dots}, 81. Jahrestagung der DPG und DPG-Frühjahrstagung, Dresden (Germany), 2017. 
	
%	\item[4.] P. Steindl, P. Klenovský, D. Geffroy, \textit{Multi-excitonic structure of type-II quantum dots}, 16$^{\mathrm{th}}$ IUVSTA International Summer School on Physics at Nanoscale, Milovy (Czech Republic), 2017.	
	
%	\item[5.] P. Steindl, P. Klenovský, D. Geffroy, \textit{Multi-excitonic structure of type-II quantum dots}, 46$^{\mathrm{th}}$ International School \& Conference on the Physics of Semiconductors ``Jaszowiec 2017'', Szczyrk (Poland), 2017. 
	
%	\item[6.] P. Steindl, P. Klenovský, E.~M. Sala, B. Alén, D. Bimberg, \textit{Identification of individual transitions in InGaAsSb/GaP quantum dot by power and temperature dependent photoluminescence}, 82. Jahrestagung der DPG und DPG-Frühjahrstagung, Berlin (Germany), 2018.
	
	
%	\item[7.] P. Steindl, P. Klenovský, E.~M. Sala, B. Alén, D. Bimberg, \textit{Recombination dynamics of InGaAsSb/GaP quantum dots}, 82. Jahrestagung der DPG und DPG-Frühjahrstagung, Berlin (Germany), 2018.
	
%\end{itemize}
\newpage 



\end{document}